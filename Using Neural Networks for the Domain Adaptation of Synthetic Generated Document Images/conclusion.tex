%\noindent
\justifying
\setlength{\parskip}{1em}




This method of unsupervised domain adaptation helps improve the performance of machine learning models in the presence of a domain shift. It enables training of models that are performant in diverse scenarios, by lowering the cost of data capture and annotation required to excel in areas where ground truth data is scarce or hard to collect.








Second, to reduce model oscillation [15], we follow
Shrivastava et al.’s strategy [46] and update the discriminators
using a history of generated images rather than the
ones produced by the latest generators. We keep an image
buffer that stores the 50 previously created images.



Neural networks are a breakthrough technique in the advancement of modern machine learning systems. However, despite the exceptional learning capacity and improved generalizability, these neural networkd still suffer from poor transferability. This is the challenge of domain adaptation — a transformation in the relationship between data collected across different domains 